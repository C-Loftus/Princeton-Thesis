@inproceedings{10.1145/2934872.2959054,
author = {Amar, Yousef and Haddadi, Hamed and Mortier, Richard},
title = {Privacy-Aware Infrastructure for Managing Personal Data},
year = {2016},
isbn = {9781450341936},
publisher = {Association for Computing Machinery},
address = {New York, NY, USA},
url = {https://doi.org/10.1145/2934872.2959054},
doi = {10.1145/2934872.2959054},
abstract = {In recent times, we have seen a proliferation of personal data. This can be attributed not just to a larger proportion of our lives moving online, but also through the rise of ubiquitous sensing through mobile and IoT devices. Alongside this surge, concerns over privacy, trust, and security are expressed more and more as different parties attempt to take advantage of this rich assortment of data.The Databox seeks to enable all the advantages of personal data analytics while at the same time enforcing **accountability** and **control** in order to protect a user's privacy. In this work, we propose and delineate a personal networked device that allows users to **collate**, **curate**, and **mediate** their personal data.},
booktitle = {Proceedings of the 2016 ACM SIGCOMM Conference},
pages = {571–572},
numpages = {2},
keywords = {Personal Data, Networks, Privacy},
location = {Florianopolis, Brazil},
series = {SIGCOMM '16}
}

@INPROCEEDINGS{9660819,  author={Suzuki, Jordan and Lameh, Saba F. and Amannejad, Yasaman},  booktitle={2021 Second International Conference on Intelligent Data Science Technologies and Applications (IDSTA)},   title={Using Transfer Learning in Building Federated Learning Models on Edge Devices},   year={2021},  volume={},  number={},  pages={105-113},  doi={10.1109/IDSTA53674.2021.9660819}}

@misc{https://doi.org/10.48550/arxiv.2007.14390,
  doi = {10.48550/ARXIV.2007.14390},
  
  url = {https://arxiv.org/abs/2007.14390},
  
  author = {Beutel, Daniel J. and Topal, Taner and Mathur, Akhil and Qiu, Xinchi and Fernandez-Marques, Javier and Gao, Yan and Sani, Lorenzo and Li, Kwing Hei and Parcollet, Titouan and de Gusmão, Pedro Porto Buarque and Lane, Nicholas D.},
  
  keywords = {Machine Learning (cs.LG), Computer Vision and Pattern Recognition (cs.CV), Machine Learning (stat.ML), FOS: Computer and information sciences, FOS: Computer and information sciences},
  
  title = {Flower: A Friendly Federated Learning Research Framework},
  
  publisher = {arXiv},
  
  year = {2020},
  
  copyright = {arXiv.org perpetual, non-exclusive license}
}

@article{beutel2020flower,
  title={Flower: A Friendly Federated Learning Research Framework},
  author={Beutel, Daniel J and Topal, Taner and Mathur, Akhil and Qiu, Xinchi and Parcollet, Titouan and Lane, Nicholas D},
  journal={arXiv preprint arXiv:2007.14390},
  year={2020}
}

@article{10.1145/3524104,
author = {Qu, Youyang and Uddin, Md Palash and Gan, Chenquan and Xiang, Yong and Gao, Longxiang and Yearwood, John},
title = {Blockchain-Enabled Federated Learning: A Survey},
year = {2022},
publisher = {Association for Computing Machinery},
address = {New York, NY, USA},
issn = {0360-0300},
url = {https://doi.org/10.1145/3524104},
doi = {10.1145/3524104},
abstract = {Federated learning (FL) is experiencing fast booming in recent years, which is jointly promoted by the prosperity of machine learning and Artificial Intelligence along with the emerging privacy issues. In the FL paradigm, a central server and local end devices maintain the same model by exchanging model updates instead of raw data, with which the privacy of data stored on end devices is not directly revealed. In this way, the privacy violation caused by the growing collection of sensitive data can be mitigated. However, the performance of FL with a central server is reaching a bottleneck while new threats are emerging simultaneously. There are various reasons, among which the most significant ones are centralized processing, data falsification, and lack of incentives. To accelerate the proliferation of FL, blockchain-enabled FL has attracted substantial attention from both academia and industry. A considerable number of novel solutions are devised to meet the emerging demands of diverse scenarios. Blockchain-enabled FL provides both theories and techniques to improve the performances of FL from various perspectives. In this survey, we will comprehensively summarize and evaluate existing variants of blockchain-enabled FL, identify the emerging challenges, and propose potentially promising research directions in this under-explored domain.},
note = {Just Accepted},
journal = {ACM Comput. Surv.},
month = {feb},
keywords = {Attacks, Blockchain, Countermeasures., Federated Learning}
}

@article{10.1145/3533708,
author = {Joshi, Madhura and Pal, Ankit and Sankarasubbu, Malaikannan},
title = {Federated Learning for Healthcare Domain - Pipeline, Applications and Challenges},
year = {2022},
publisher = {Association for Computing Machinery},
address = {New York, NY, USA},
issn = {2691-1957},
url = {https://doi.org/10.1145/3533708},
doi = {10.1145/3533708},
abstract = {Federated learning is the process of developing machine learning models over datasets distributed across data centers such as hospitals, clinical research labs, and mobile devices while preventing data leakage. This survey examines previous research and studies on federated learning in the healthcare sector across a range of use cases and applications. Our survey shows what challenges, methods, and applications a practitioner should be aware of in the topic of federated learning. This paper aims to lay out existing research and list the possibilities of federated learning for healthcare industries.},
note = {Just Accepted},
journal = {ACM Trans. Comput. Healthcare},
month = {apr},
keywords = {federated learning, transfer learning, GDPR}
}

@inproceedings{10.1145/3534678.3539039,
author = {Zhang, Qi and Wu, Tiancheng and Zhou, Peichen and Zhou, Shan and Yang, Yuan and Jin, Xiulang},
title = {Felicitas: Federated Learning in Distributed Cross Device Collaborative Frameworks},
year = {2022},
isbn = {9781450393850},
publisher = {Association for Computing Machinery},
address = {New York, NY, USA},
url = {https://doi.org/10.1145/3534678.3539039},
doi = {10.1145/3534678.3539039},
abstract = {Felicitas is a distributed cross-device Federated Learning (FL) framework to solve the industrial difficulties of FL in large-scale device deployment scenarios. In Felicitas, FL-Clients are deployed on mobile or embedded devices, while FL-Server is deployed on the cloud platform. We also summarize the challenges of FL deployment in industrial cross-device scenarios (massively parallel, stateless clients, non-use of client identifiers, highly unreliable, unsteady and complex deployment), and provide reliable solutions. We provide the source code and documents at https://www.mindspore.cn/. In addition, the Felicitas has been deployed on mobile phones in real world. At the end of the paper, we demonstrate the validity of the framework through experiments.},
booktitle = {Proceedings of the 28th ACM SIGKDD Conference on Knowledge Discovery and Data Mining},
pages = {4502–4509},
numpages = {8},
keywords = {data mining under privacy constraints, large-scale, cross-device federated learning, distributed framework},
location = {Washington DC, USA},
series = {KDD '22}
}
@inproceedings{10.1145/3529399.3529416,
author = {Marinus Dieperink, Matthijs and Achoura Ho, Romy and Theodorus Johannes Koomen, Leonardus and Willem Hubertus Keuzenkamp, Joep and Hendrik Schot, Jonathan and Pijl, Tim},
title = {Federated Learning in the Bubl Platform to Enhance the Privacy of Personal Patient Data},
year = {2022},
isbn = {9781450395748},
publisher = {Association for Computing Machinery},
address = {New York, NY, USA},
url = {https://doi.org/10.1145/3529399.3529416},
doi = {10.1145/3529399.3529416},
abstract = {Data privacy and security are currently an important societal topic that rightfully garners much attention. In an effort to make people the owners of their personal data, the Bubl platform will provide its users with a secure personal data vault. This platform will have a focus on medical and healthcare data, because of its sensitive nature. Gathering important insights from users’ data could be useful, but due to the high privacy and security standards required by Bubl, it becomes impossible to deploy standard Machine Learning (ML) techniques. These methods require centralization of all training data, which is not allowed. This problem can be solved using techniques from the research field of Privacy Preserving Machine Learning (PPML). Therefore, a system to facilitate PPML within the Bubl platform is developed. More specifically, we employ a technique called Federated Learning (FL). In our FL implementation in the Bubl platform, we focus on minimizing Random-Access Memory (RAM) usage to adhere to the constraints posed by the small computational budgets of the data vaults. Challenges that arise are non-Independent and Identically Distributed (IID) data and the fact that patient vaults contain very few data samples. The latter is the main focus of this research as it is underdeveloped in the FL literature. Currently, we are still working on acquiring the results which are expected in the coming months. At this moment, only preliminary results are discussed that reflect on the effect of the number of clients and the distribution of non-IID data on the ML performance.},
booktitle = {2022 7th International Conference on Machine Learning Technologies (ICMLT)},
pages = {99–104},
numpages = {6},
keywords = {Federated Averaging, Artificial Intelligence, Machine Learning},
location = {Rome, Italy},
series = {ICMLT 2022}
}

% Youtube Videos
https://www.youtube.com/watch?v=nBGQQHPkyNY
https://www.youtube.com/watch?v=dcm8fjBfro8

% Github Repositories
https://github.com/adap/flower/tree/main/examples/quickstart_pytorch
